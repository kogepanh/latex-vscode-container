%----------------------------------------------------------------------
% 要変更
\documentclass[a4paper,11pt]{ujreport} % Docker(VSCode) 用
% \documentclass[a4paper,11pt]{jreport} % Cloud LaTex 用(pLaTex)
%----------------------------------------------------------------------

%----------------------------------------------------------------------
% 変更不要
\usepackage{style/nislab,style/thesis}
%----------------------------------------------------------------------

%----------------------------------------------------------------------
% 修論・卒論 執筆 注意事項
%----------------------------------------------------------------------
% 章立て
% * 章のなかに節が1つだけの場合は,節を立てる必要がありません.節の中に小節が1つだけの場合も同様です.
% * つまり,chapterの中には,sectionは2つ以上あるべきで,sectionが1つの場合は,そのsectionは不要で,chapter直下に内容を記述すべきです.
% * 論文の概要と「はじめに」(イントロダクション)は異なるものです.概要は結果も含めた論文全体の概要であり,「はじめに」は解決すべき問題を明示するための背景情報および論文でその問題をどう取り扱うかの導入について記載するものです.

% 背景と目的
%   * 最初に「背景と目的」を書くのが一般的かと思いますが,「背景と目的」は論文の概要ではありません.
%   * なぜ,この研究をする必要があるのか,一般的な世間の状況と,研究を行う必要性を書くことになります.
%   * 単に今までに無かったので作ったではダメです(例:AとBをつなぐシステムが無かったので作りました)
%   * 世の中でその問題にどう取り組んでいるかは,一般には「関連研究」のところで説明します.
%   * 一般的な背景情報はこの章にまとめてください(本文中に一般的背景情報を書くべきではありません)
%   * また,この章では,論文の技術的な内容や結果を書く必要がありません.なぜなら「背景と目的」だからです.
%   * 一方,論文の「概要」は,研究結果および論文の結論まで含めて書くべきです.

% 提案方式の説明
%   * 問題に関して,自分の解決方法を説明します.
%   * 問題そのものを簡単に理解できない場合は,問題についても詳解が必要です.
%   * 問題をどうやって解決するかを手順を追って解説します.
%   * 複数の問題が関連している場合は問題を分離して説明します.
%   * 自分の解決方法が従来とどこが違いどう工夫しているかを明記する必要があります.

% 評価
%   * 自分の解決方法を問題点に適応してどういう結果が得られたかについて説明をします.
%   * 従来技術や手法と比較してどこがどうよくなったかを示します.
%   * どのような環境で比較したかを説明する必要があります.
%   * 定量的に以前(関連研究)と比べてこうよくなったと説明できればベターです.

% 考察
%   * なるべくなら考察を章で分けて下さい.提案方式に関する「評価」に関して記載する章があるのが一般的かと思いますが,「考察」の章では,従来技術や関連研究と比較して,評価結果がどうであったかを考察して下さい.

% まとめ (★特に修論の場合の注意点★)
%   * 800 字から1000字ぐらいでまとめてください.
%   * 背景,解決すべき問題点,提案内容,結果,考察,研究の意義などを含めて記載してください.
%   * 結果については,過去形(・・・実施した.・・・評価した.・・・確認した.など)で記載してください.

% 参考文献
%   * 勉強した書籍を列挙するものではありません.
%   * 本文中に引用した技術などを記載するものです.
%   * 参考文献の番号は,必ず本文中の引用場所を示します.(本文中に引用の番号がない参考文献は存在しません)

% 表現
%   * 自分が出した結果に対して「・・・と考えられる」や「・・・と言える」は使わないで下さい.
%   * 関連研究などの他人の出した結論に対しては OK.
%   * 一般的に断言できない場合は,条件を設定して「・・・という条件においては,・・・である」と断定してください.
%   * 文中に副詞を用いる場合は,その副詞が本当に必要かどうかをよく考えて下さい.
%   * 略語は,論文の最初に登場したところで何の略語であるかを記載して下さい.
%   * 文中は定量的な表現を使って下さい.大きい/小さい,長い/短い,速い/遅い,など.何をもって大きいと言えるのか,などを考えて下さい.

% ページ数
%   * 修論は本文が20ページ以内.図を含めて20ページを越えても問題ありません.
%   * 卒論は20ページ以上で記載して下さい.
%   * 出版物になりますので,権利関係が明確になっていない場合,同志社大学あるいはその関係者以外に著作権のある図の利用は不可です.

% 提出
%   * 必ず,締切の前日までに事務に提出してください.締切の当日に(交通機関の遅延,病気などの一般的には正当な理由があっても)遅れた場合は受理されません.

%----------------------------------------------------------------------
% 表紙用
%----------------------------------------------------------------------
\type{1}  % 1:卒業論文 2:修士論文
\title{タイトルタイトルタイトルタイトル\\タイトルタイトルタイトル}  % 日本語タイトル
\etitle{Title Title Title Title Title Title Title Title Title Title \\Title Title Title}  % 英語タイトル
\author{研究 太郎}  % 著者
\date{2022年2月31日}  % 日付
\advisor{佐藤 健哉 教授}  % 指導教員
\university{同志社大学} % 大学名
\department{理工学部 情報システムデザイン学科} % 専攻
\lab{ネットワーク情報システム研究室}  % 研究室
\entryyear{2018}  % 入学年度
\studentnumber{0001}  % 学生番号

%----------------------------------------------------------------------
% 変更不要
%----------------------------------------------------------------------
\begin{document}
\maketitle
\clearpage

%----------------------------------------------------------------------
% 概要
%----------------------------------------------------------------------
% 卒業論文は日本語(200~400文字),修士論文は英語(200~300単語)で書く.
% 改行は不要
%----------------------------------------------------------------------
\begin{abstract}
  概要は,論文全体を読まなくてもその研究の序論から結論までが理解できるようにするものです.
  本文の内容を忠実に反映させるだけでなく,研究の新規性や重要性を簡潔かつ的確に伝えられることが,より多くの読者を獲得する鍵となります.
  概要は,研究目的から研究方法,研究結果,そして結論に至る肝心な要素のすべてが要約されていなければならないのです.
\end{abstract}

% キーワードを3つ設定する
% 卒業論文は日本語,修士論文は英語
\addkeywords{赤}{青}{黄}

%----------------------------------------------------------------------
% 変更不要
%----------------------------------------------------------------------
\footnote[0]{本論文に掲載の製品名・会社名等は,一般にそれぞれの会社の商標,または登録商標である.}
\footnote[0]{なお,本文中では\texttrademark ・ \textregistered 等のマークは特に明記していない.}

%----------------------------------------------------------------------
% 変更不要
%----------------------------------------------------------------------
% ページ番号をギリシャ数字にする
\pagenumbering{roman}

% 目次を1ページから始めるために表紙を0ページにする
\setcounter{page}{0}

% 目次を作成
\tableofcontents

% 改ページ
\clearpage

% 以降,ページ番号をアラビア数字で振り直す
\pagenumbering{arabic}

%----------------------------------------------------------------------
% はじめに
%----------------------------------------------------------------------
\chapter{はじめに}
\label{chap:Introduction}

最初はイントロ的なことを書く.
各段落の終わりには \verb|\par| を書くようにしてください.\par

\section{現状と問題点}
\label{sec:現状と問題点}

最近の現状と問題点とか.\par

\section{数式の書き方}
\label{sec:数式の書き方}

アインシュタイン方程式は以下の通りである.\par

\begin{equation}
  R_{\mu\nu} - \frac{1}{2} g_{\mu\nu} R = \frac{8\pi G}{c^2} T_{\mu\nu}
\end{equation}

ベクトルの書き方は以下の通り.\par

\begin{itemize}
  \item 普通の$\alpha$は\verb|\alpha|で書く。
  \item \verb|$\vec{\alpha}$| で $\vec{\alpha}$
  \item \verb|\usepackage{bm}| している場合は
        \verb|$\bm{\alpha}$| で $\bm{\alpha}$
  \item 並べると,$\alpha$, $\vec{\alpha}$, $\bm{\alpha}$
\end{itemize}

%----------------------------------------------------------------------
% つぎに
%----------------------------------------------------------------------
\chapter{つぎに}
\label{chap:つぎに}

この辺から本番.\par

\section{文献の引用の仕方}
\label{sec:分権の引用の仕方}

このように参考文献\cite{LaTexWiki,渡辺豊2016}を書きます.
複数の文献を参照する場合は, \verb|\cite{aaa},\cite{bbb}| と書くのではなく, \verb|\cite{aaa,bbb}| と参照してください.\par

\section{図の挿入の仕方}
\label{sec:図の挿入の仕方}

図は以下のように挿入し,\figref{fig:sample1}と引用します.
図の位置は基本的に \verb|[tb]| にして, \verb|[!htb]| などは使わないでください.
また, \verb|begin{center} ... end{center}| ではなく, \verb|\centering| を使うようにしてください.\par

図の大きさを変えたい場合は, \verb|\includegraphics[width=0.8\linewidth]{...}| のように書くことができます.\par

\begin{figure}[tb]
  \centering
  \includegraphics[width=\linewidth]{img/sample1.pdf}
  \caption{サンプル画像1}
  \label{fig:sample1}
\end{figure}

%----------------------------------------------------------------------
% おわりに
%----------------------------------------------------------------------
\chapter{おわりに}
\label{sec:Conclusion}

まとめを書きましょう.
800字から1000字ぐらいでまとめてください.
背景,解決すべき問題点,提案内容,結果,考察,研究の意義などを含めて記載してください.
結果については,過去形(・・・実施した.・・・評価した.・・・確認した.など)で記載してください.\par

\clearpage

%----------------------------------------------------------------------
% 謝辞
%----------------------------------------------------------------------
\chapter*{謝辞}
\addcontentsline{toc}{chapter}{謝辞}  % 章番号のない章を目次に表示させる
\label{sec:Acknowledgments}

謝辞には章番号をつけなくてもよいので,\verb|\chapter*{}| という具合に書きます.

\clearpage

%----------------------------------------------------------------------
% 付録
%----------------------------------------------------------------------
\appendix
\chapter{ソースコード}
\label{apndx:src}

プログラム文とかを書きたい場合は,以下のようにしてみます.\verb|\usepackage{ascmac}|して\verb|screen| 環境を使うと,枠がつきます.

\begin{screen}\begin{verbatim}
#include <iostream>
using namespace std;

int main() {
  for(int i = 1; i <= 5; i++) {
    cout << "こんにちは, C++ の世界! " << i << endl;
  }
  return 0;
}
\end{verbatim}\end{screen}

\clearpage

%----------------------------------------------------------------------
% 参考文献
%----------------------------------------------------------------------
\renewcommand{\bibname}{参考文献}
\addcontentsline{toc}{chapter}{参考文献}  % 章番号のない章を目次に表示させる

% thebibliography を利用する場合は以下を使用(拘りがなければこちらでOK)
\begin{thebibliography}{99}
  \bibitem{LaTexWiki} Latex Wiki. \url{https://texwiki.texjp.org/}.
  \bibitem{渡辺豊2016} 渡辺 豊, "角皆静男先生のご逝去を悼む", 地球化学, vol.50, no.1, pp.1-3, 2016.
\end{thebibliography}

% BibTex を利用する場合は以下を使用(初めての人には難しいかも)
% \bibliographystyle{junsrt}
% \bibliography{myref}

\clearpage

%----------------------------------------------------------------------
% 研究業績
%----------------------------------------------------------------------
\renewcommand{\bibname}{研究業績}
\addcontentsline{toc}{chapter}{研究業績}

\begin{thebibliography}{99}
  \bibitem{} 研究太郎,研究次郎,研究三郎,研究の研究による研究のための研究,第1回研究フォーラム,vol.1,pp.1-10,2020.
  \bibitem{} 研究太郎,研究次郎,研究三郎,研究の研究による研究のための研究の応用,研究論文誌,vol.1,pp-1-12,2020.
\end{thebibliography}

\clearpage

%----------------------------------------------------------------------
\end{document}
%----------------------------------------------------------------------
